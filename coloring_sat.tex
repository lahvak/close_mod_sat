\documentclass[letterpaper]{article}
\usepackage{pythontex}
\usepackage[]{amsmath}
\usepackage{amssymb}
\usepackage{acronym}
\usepackage[]{graphicx}
\usepackage[margin=1.2in]{geometry}
\graphicspath{{./pythontex-files-coloring_sat/}}

\acrodef{CNF}{Conjunctive Normal Form}

\let\vel\vee
\let\at\wedge

\title{Generating Closed Modular Colorings of Graphs Using a SAT Solver}

\author{Jan Hlavacek and Garry L.~Johns\\Saginaw Valley State University}

\begin{document}
\maketitle
\section{Introduction}
We start with a graph $G$.  To keep things simple, we will assume
that $G$ has no true twins.  We color all vertices with colors from $0$ to $k-1$.  Then we
calculate the closed modular coloring:  for each vertex you add the colors of
all its neighbours, add it to its color, and assign that number mod $k$ as its
new color. 

A coloring is proper if no two adjacent vertices have the same color. 

We are trying to see if it is possible to color a graph $G$ so that the
resulting closed modular coloring is proper. 

\section{Description using logic}

We will have $k$ colors, $0,1,\cdots,k-1$.  We try to describe the condition of
having a closed modular coloring using logic formulas, so we can
use a SAT-solver to find out if the condition is satisfiable for
the graph G. We will need to introduce a number of logical
variables and write a logic formula using these variables that
will be equivalent to $G$ having a proper closed modular coloring.

\subsection{Vertices}

\subsubsection{Coloring a vertex}

For each vertex $v$, there will be $k$ color variables $c_v^i$, $i =
0,\dots,k-1$.  Exactly one of those will be true at a time, since a vertex cannot
have more than one color:
\begin{gather}
   c_v^1 \vel c_v^2 \vel c_v^3 \vel \dots \vel c_v^k \label{eq:colors1}\\
   \neg c_v^i \vel \neg c_v^j \text{ for $i \neq j$\label{eq:colors2}}
\end{gather}

\subsubsection{Modular coloring}

Now we need to figure out what will be the new color of the vertex in the
modular coloring.  That will depend on the neighbouring vertices, as well as on
the vertex itself.

We will take a list of all the neighbours of the vertex $v$, including the
vertex $v$ itself, and go through all
the possible colorings of this list.  Each such coloring $C_s$ will be assigned
a variable $n_v^s$, that will be equivalent to all the neighbouring vertices
having exactly the colors from $C_s$. 

For example, if the list of neighbours of $v$ is $[v,w,y,x,z]$, then for coloring
$C_s = [2,1,3,2,4]$ of those vertices the variable $n_v^s$ will satisfy
\begin{equation}
 n_v^s \equiv c_v^2 \at c_w^1 \at c_y^3 \at c_x^2 \at c_z^4
   \label{eq:neighbours}
\end{equation}

For each color $j$, there will be possibly several colorings $C_{s_1}, C_{s_2},
\dots, C_{s_{t_j}}$ of the neighbouring vertices that will produce $j$ as the
modular coloring of the vertex $v$.  We introduce $k$ variables $m_v^j$ for $j
= 1, 2, \dots, k$, satisfying
\begin{equation}
   m_v^j \equiv n_v^{s_1} \vel n_v^{s_2} \vel \cdots \vel n_v^{s_{t_j}}
   \label{eq:modular}
\end{equation}

\subsection{Edges}

The rules for edges will be simple enough.  All we need to assure is that the
resulting modular coloring is proper, that is for each edge $(v,w)$, the
modular coloring of the vertex $v$ is different than the modular coloring of
the vertex $w$.   That means that for each color $j = 0,\dots,k-1$, $m_v^j$ and
$m_w^j$ cannot be both true:
\begin{equation}
   \neg m_v^j \vel \neg m_w^j \text{ for $j = 0,\dots,k-1$}
   \label{eq:edge}
\end{equation}

\section{Translation to CNF}

Satisfiability solvers typically require their input to be in the \ac{CNF}, that is
a written as a conjunction of clauses, where each clause is a disjunction of
single variables or their negations.  We see that equations \eqref{eq:colors1},
\eqref{eq:colors2} and \eqref{eq:edge} already are in \ac{CNF}, while
\eqref{eq:neighbours} and \eqref{eq:modular} need to be translated. 

The formula $a \equiv (b \at c)$ can be translated into \ac{CNF} as 
\begin{equation}
   \left(c \vee \neg a\right) \wedge \left(b \vee \neg a\right) \wedge \left(a \vee \neg b \vee \neg c\right)
   \label{eq:trans1}
\end{equation}
while the formula $a \equiv (b \vel c)$ can be translated to 
\begin{equation}
   \left(a \vee \neg c\right) \wedge \left(a \vee \neg b\right) \wedge \left(b \vee c \vee \neg a\right)
   \label{eq:trans2}
\end{equation}

Equation \eqref{eq:neighbours} will then translate to 
\begin{gather}
   \label{eq:neighbour_trans_1} (c_v^2 \vel \neg n_v^s) \at (c_w^1 \vel \neg n_v^s) \at (c_y^3 \vel \neg
   n_v^s) \at (c_x^2 \vel \neg n_v^s) \at (c_z^4 \vel \neg n_v^s)\\
   \label{eq:neighbour_trans_2} n_v^s \vel \neg c_v^2 \vel \neg c_w^1 \vel \neg c_y^3 \vel \neg
   c_x^2 \vel \neg c_z^4
\end{gather}
In general, for each coloring $C_s$ of the neighbours of vertex $v$, we will have the conjunction of 
\begin{equation}
   c_w^j \vel \neg n_v^s
   \label{eq:general_neighbour_1}
\end{equation}
where $w$ runs through all the neighbors of $v$, including $v$, and for each $w$, $j$ is the
color of $w$ in the coloring $C_s$, with 
\begin{equation}
   n_v^s \vel \neg c_{w_1}^{j_1} \vel \neg c_{w_2}^{j_2} \vel \cdots
   \label{eq:general_neighbour_2}
\end{equation}
where $w_i$ are all the neighbors of $v$, including $v$, and $j_i$ are their corresponding
colors in $C_s$. 

The formula \eqref{eq:modular} will translate to 
\begin{gather}
   \label{eq:modular_general_1} m_v^j \vel \neg n_v^{s_i}\\
   \label{eq:modular_general_2} n_v^{s_1} \vel n_v^{s_2} \vel \cdots \vel
   n_v^{s_{t_j}} \vel \neg m_v^j
\end{gather}
where $j$ runs through all the colors $1$ to $k$, and $s_i$ runs through all
indices $s$ such that the corresponding coloring $C_s$ of all the neighbours of
$v$ will produce the modular coloring of $v$ with the color $j$. 

\section{Implementation}

\subsection{Imports and Setup}

We will use the \pyv|pycosat| module, which provides a Python interface to the
picosat SAT solver.  We will also need few functions from the \pyv|itertools|
module. We will use the \pyv|graph_tool| module to handle the graph
stuff. 

\begin{pyblock}
import pycosat
from itertools import count, islice, product
import graph_tool as gt
\end{pyblock}

The \pyv|pycosat| module requires variables to be represented by integers. Each
variable will have to be associated with a unique integer.  To keep track which
integers we already used, we will use the \pyv|count()| function from
the \pyv|itertools| module. 

\subsection{Assigning Variables}

Given a graph, we will need to assign the initial variables for original colors
($c_v^j$) and for colors of a modular coloring ($m_v^j$).  We need to create two property maps for the
given graph, one for original colors, and one for the closed modular coloring.
Each property will be a vector of integers of length $k$, where $k$ is the
number of colors.  The following function will create and populate one such map

\begin{pyblock}
def assign_color_vars(g, no_of_colors, vars_generator):
   """
   Create property map for graph g, containing variable numbers 
   for colors.

   Args:
      g: A graph
      no_of_colors: the number of colors used for graph coloring.
      vars_generator: generator that produces numbers for logical
         variables, to be used in pycosat clauses.

   Returns:
      A vertex property map of 'vector<int>' type, filled with 
      numbers of pycosat variables corresponding to possible 
      colors for each vertex.
   """
   colors_map = g.new_vertex_property("vector<int>")
   for v in g.vertices():
      colors_map[v] = list(islice(vars_generator, no_of_colors))
   return colors_map
\end{pyblock}

We will call this function twice for each graph, first time to generate
variable numbers for all $c_v^j$ variables, the second time to generate numbers
for all $m_v^j$ variables. 

\subsection{Generating Clauses}

Now we will write some functions that will generate clauses for the SAT solver.
Each clause will be a list of positive or negative integers.  Positive number
means that the variable with the number appear in the conjunction, while a
negative number means that the negation of the variable with the opposite
number appears in the conjunction. 

\subsubsection{Clauses from equations \eqref{eq:colors1} and \eqref{eq:colors2}}

These clauses are grouped by individual vertices.  We will write a function
that will receive a vertex and generate all these clauses for that vertex.  For
the equation \eqref{eq:colors2}, we will only need the clauses for $j < i$,
since disjunction is commutative.

\begin{pyblock}
def vertex_color_clauses(v, color_map):
   """
   Return a list of clauses dealing with individual vertex 
   initial color.

   Args:
      v: a graph vertex
      color_map: a vertex property map that assigns variable 
      numbers to colors of individual vertices.

   Returns:
      A list of lists, each of them representing one clause.
   """
   l = []
   colors = color_map[v]
   l.append(list(colors))
   for i in range(1,len(colors)):
      for j in range(i):
         l.append([-colors[i],-colors[j]])
   return l
\end{pyblock}

\subsubsection{Clauses from equation \eqref{eq:edge}}

There will be a group of clauses for each edge, assuring that the resulting
modular coloring is proper.  The function will receive an edge, and return a
list of clauses. It will need the lists of modular coloring colors for each end
of the edge, and create a list of clauses assuring that the colors on the ends
of the edge are not the same.  

\begin{pyblock}
def edge_clauses(e, modular_color_map):
   """
   Return a list of clauses assuring that the colors of the 
   modular coloring on the ends of the edge are different.

   Args:
      e: an edge of a graph
      modular_color_map: a vertex property map that assigns 
      variable numbers to colors in modular coloring of 
      individual vertices.

   Returns:
      A list of lists, each of which correspond to one clause.
   """
   l = []
   mc1 = modular_color_map[e.source()]
   mc2 = modular_color_map[e.target()]
   for i,j in zip(mc1,mc2):
      l.append([-i,-j])
   return l
\end{pyblock}

\subsubsection{Clauses from equations \eqref{eq:general_neighbour_1},
\eqref{eq:general_neighbour_2}, \eqref{eq:modular_general_1} and
\eqref{eq:modular_general_2}}

This is the trickiest part.  For a given vertex $v$, we take the list of all
the neighbours of $v$, including $v$.  Then we will go through all the possible colorings of
this list, for each coloring we allocate a new variable, and make it equivalent
to a conjunction of the vertex color variables corresponding to the colors in
the coloring.  At the same time, we will check what color would a modular
coloring produced by this coloring of neighbours of $v$ produce at $v$.   We
will add the variable we allocated to this coloring to a list of variables for
the obtained modular color. At the end we will go through all the modular color
variables for vertex $v$, and generate clauses that will make each equivalent
to the disjunction of all the variables added to their lists.

We will start with two auxiliary functions.  Each will receive a variable
number as the first argument, and a list of variable numbers as its second
argument.  The first will generate clauses making the first variable equivalent
to the conjunction of all the variables in the list, the second one will do the
same for a disjunction of all the variables on the list. 

\begin{pyblock}
def equiv_to_conjunction(a,vars):
   """
   Return a list of clauses that assure that the variable a is 
   equivalent to the conjunction of all the variables in the list 
   vars.

   Args:
      a: integer, a number of a variable
      vars: a list of integers, each corresponding to a variable

   Returns:
      a list of lists of integers.  Each list of integers 
      corresponds to one clause.  The conjunction of all the 
      clauses will be a CNF version of the equivalence of the 
      variable a with the conjunction of variables in l.
   """
   l = [[a] + [-b for b in vars]]
   for b in vars:
      l.append([b,-a])
   return l

def equiv_to_disjunction(a,vars):
   """
   Return a list of clauses that assure that the variable a is 
   equivalent to the disjunction of all the variables in the list 
   vars.

   Args:
      a: integer, a number of a variable
      vars: a list of integers, each corresponding to a variable

   Returns:
      a list of lists of integers.  Each list of integers 
      corresponds to one clause.  The conjunction of all the 
      clauses will be a CNF version of the equivalence of the 
      variable a with the disjunction of variables in l.
   """
   l = [[-a] + vars]
   for b in vars:
      l.append([-b,a])
   return l
\end{pyblock}

The next function will receive a vertex $v$, and produce all the clauses from
equations \eqref{eq:general_neighbour_1}, \eqref{eq:general_neighbour_2},
\eqref{eq:modular_general_1} and \eqref{eq:modular_general_2}. 

\begin{pyblock}
def modular_coloring_clauses(v,color_map,modular_color_map,
                             vars_generator):
   """
   Return the list of clauses that will assure that the colors in 
   the modular_coloring_map are created as a closed modular 
   coloring from the colors in the color_map.

   Args:
      v: a vertex of a graph
      color_map: a vertex property map that assigns variable 
         numbers to colors of individual vertices.
      modular_color_map: a vertex property map that assigns 
         variable numbers to colors in modular coloring of 
         individual vertices.
      vars_generator: generator that produces numbers for logical
         variables, to be used in pycosat clauses.

   Returns:
      A list of lists, each of which correspond to one clause.

   Notes:
      May allocate new pycosat variables using the vars_generator.
   """
   modular_colors = list(modular_color_map[v])
   k = len(modular_colors) # number of colors
   m_lists = {m:[] for m in modular_colors}
   neighbours = list(v.out_neighbours()) + [v]
   colorings = product(range(k),repeat=len(neighbours))
   l = []
   for cs in colorings:
      new_var = next(vars_generator)
      modular_color = sum(cs) % k
      m_lists[modular_colors[modular_color]] += [new_var]
      neighbour_color_vars = []
      for w,c in zip(neighbours,cs):
         neighbour_color_vars.append(color_map[w][c])
      l += equiv_to_conjunction(new_var,neighbour_color_vars)

   for m,s in m_lists.items():
      l += equiv_to_disjunction(m,s)

   return l
\end{pyblock}

\subsubsection{Putting all clauses together}

Now we need to put all the clause generating functions together. The following
function will take a graph and a number of colors, and generate all the clauses
for the graph.  The result can then be fed directly to the pycosat solver. 

The steps will be:
\begin{enumerate}
   \item Create a new variable generator.
   \item Assign color and modular color variables to all vertices.
   \item For each vertex:
      \begin{enumerate}
         \item Generate all the clauses for equations \eqref{eq:colors1} and
            \eqref{eq:colors2}.
         \item Generate all the clauses for equations
            \eqref{eq:general_neighbour_1}, 
            \eqref{eq:general_neighbour_2}, 
            \eqref{eq:modular_general_1} and 
            \eqref{eq:modular_general_2}.
      \end{enumerate}
   \item For each edge, generate all clauses for equation \eqref{eq:edge}.
   \item Return a complete list of all the generated clauses.  Also return the
      two vertex property maps, since we will need them to decode the result.
\end{enumerate}

\begin{pyblock}
def graph_clauses(g,num_of_colors):
   """
   Return the complete lists of all the clauses describing 
   conditions for a proper closed modular coloring of g.

   Args:
      g: a graph_tool graph
      num_of_colors: number of colors for the modular coloring

   Returns: 
      a list of lists of integers: each list of integers
      corresponds to a disjunction of variables or their
      negations, and two vertex property maps containing 
      the colors for the original coloring and the corresponding 
      modular coloring.
   """
   l = []
   vars_generator = count(1)
   color_map = assign_color_vars(g,num_of_colors,vars_generator)
   modular_color_map = assign_color_vars(g,num_of_colors,
                                         vars_generator)

   for v in g.vertices():
      l += vertex_color_clauses(v,color_map)
      l += modular_coloring_clauses(v,color_map,modular_color_map,
                                    vars_generator)

   for e in g.edges():
      l +=  edge_clauses(e, modular_color_map)

   return l, color_map, modular_color_map
\end{pyblock}

\subsection{Making sense of the result}

The pycosat solver returns a long list of positive or negative integers, one
for each variable.  Each number corresponds to one variable, and is positive if
the corresponding variable is true, and negative if the variable is false.  We
are only interested in the variables on the beginning, the ones that correspond
to the color variables and modular color variables for each vertex. Out of
these, we only want to know, for each vertex, which of its color variables and
which of its modular color variables has a positive number in the result. 

A safe way (see below for some ideas that could make things easier if we could
rely on some assumptions) would be:
\begin{enumerate}
   \item Discard all negative entries from the list
   \item Create a set from the positive entries, to make lookup faster
   \item Iterate through vertices of the graph:
      \begin{enumerate}
         \item For each vertex, iterate through the colors, asking if each is
            in the set, until you find one
         \item Do the same for modular colors
      \end{enumerate}
   \item For each vertex, remember the found color and modular color.
\end{enumerate}

If we could assume that:
\begin{itemize}
   \item Iteration through vertices of a graph always happens in the same
      order. (that is probably true)
   \item The variables returned by pycosat solver always come ordered by
      absolute value. (that is also probably true.  We could also sort the list
      ourselves)
   \item The first (number of colors)*(number of vertices) variables correspond
      to the colors, the next to modular colors (that is true, but it would be
      too hackish to rely on it.  Changing one part of the code could
      completely mess up the other part, introducing messy hard to detect
      errors.)
\end{itemize}
we could simply discard all negative variables, zip together the first (number
of vertices) of the remaining numbers with the vertices of the graph to get the
colors, and zip together the next (number of vertices) of the numbers with the
vertices to get the modular coloring. 

\begin{pyblock}
def extract_colors(g,vertex_map,result_list):
   """
   Extract colors of each vertex from the result list.

   Args:
      g: a graph_tool graph
      vertex_map: a vertex property map.  Each vertex has a list
         of its color variables. Could be original color map or
         modular map.
      result_list: the list of variable values returned by the
         pycosat solver.

   Returns:
      a new vertex property map: one integer for each vertex, the
      color of the vertex in the coloring found by the solver.
   """
   s = set(i for i in result_list if i > 0)
   coloring = g.new_vertex_property('int')
   for v in g.vertices():
      for i,n in enumerate(vertex_map[v]):
         if n in s:
            coloring[v] = i
            break

   return coloring
\end{pyblock}

We will try to plot the graph to verify that the solver indeed found a coloring
with the right properties.  We can either use colors to show the original
coloring and numbers to show the modular coloring (default), or numbers to show
the original coloring and colors to show the modular coloring (set
\pyv|color_orig| to \pyv|False|).  The default setting will make it easier to
see patterns in the original coloring that lead to a closed modular coloring,
while the alternative setting will make it easier to quickly verify that the
resulting coloring is indeed a proper closed modular coloring generated by the
original coloring.

\begin{pyblock}
from graph_tool.draw import graph_draw

def plot_graph(g,coloring,modular_coloring,
               color_names,layout,
               filename,size=(600,600),
               color_orig=True):
   fcolors = g.new_vertex_property('string')
   labels = g.new_vertex_property('string')
   if color_orig:
      for v in g.vertices():
         fcolors[v] = color_names[coloring[v]]
         labels[v] = str(modular_coloring[v])
   else:
      for v in g.vertices():
         fcolors[v] = color_names[modular_coloring[v]]
         labels[v] = str(coloring[v])

   graph_draw(g,vertex_fill_color=fcolors,
              vertex_text=labels,
              pos=layout,
              output_size=size,
              output=filename)
\end{pyblock}

\section{Check the coloring}

Just to make sure, we will make an independent check if
\begin{enumerate}
   \item the produced coloring is indeed a closed modular coloring obtained
      from the original coloring.
   \item the produced coloring is proper.
\end{enumerate}

These can be done the following way:
\begin{enumerate}
   \item Loop through vertices, calculating modular coloring and comparing it
      with the assigned modular color.
   \item Loop through edges, making sure the two colors at the ends of the edge
      are different. 
\end{enumerate}

\begin{pyblock}
def is_modular(g,coloring,modular_coloring,k):
   """
   Check if the coloring described by modular_coloring is really a
   modular coloring. 

   Args:
      g: a graph-tool graph
      coloring: a vertex property map describing the original
         coloring
      modular_coloring: a vertex property map describing the
         coloring that is tested for modularity
      k: an integer, the number of colors available

   Returns:
      True if the coloring in modular_coloring is really modular,
      False otherwise
   """
   for v in g.vertices():
      neighbours = list(v.out_neighbours()) + [v]
      true_modular_color = sum(coloring[w] 
                               for w in neighbours) % k
      if modular_coloring[v] != true_modular_color:
         return False

   return True

def is_proper(g,coloring):
   """
   Check if the coloring of g described by coloring is proper

   Args:
      g: a graph-tool graph
      coloring: a vertex property map describing a coloring of g

   Return:
      True if the coloring is proper, False otherwise
   """
   for e in g.edges():
      if coloring[e.source()] == coloring[e.target()]:
         return False

   return True
\end{pyblock}

\subsection{Test a given graph}

Now we take the generated clauses and feed them into the
SAT-solver.  Then we decode the result, and return a tuple whose
first element will be \pyv|True| if we found a solution, or
\pyv|False| if we did not. 

In the case a solution was found, the second element will be the
string \pyv|'SAT'|, and the third and fourth element will be the
original coloring and the modular coloring, respectively. 

If the solution was not found, the second element of the tuple
will be a string explaining what happened:
\begin{description}
   \item[UNSAT] the solver decided that there is no solution.
   \item[UNKNOWN] the solver was not able to find a solution, and
      was not able to prove that there is none.
   \item[NOT MODULAR] the solver found a solution, but the
      supposedly modular coloring returned by the solver is not a
      closed modular coloring generated by the original coloring
      returned by the solver. This should \emph{never} happen!
   \item[NOT PROPER] the solver found a solution, but the
      supposedly modular coloring returned by the solver is not a
      proper. This should \emph{never} happen!
\end{description}

\begin{pyblock}
def g_has_pcmc(g,k):
   l, c_m, m_c_m = graph_clauses(g,k)
   result = pycosat.solve(l)
   
   if (result == 'UNSAT') or (result == 'UNKNOWN'):
      return (False,result)

   coloring = extract_colors(g,c_m,result)
   modular_coloring = extract_colors(g,m_c_m,result)

   if not is_modular(g,coloring,modular_coloring,k):
      return(False,'NOT MODULAR',coloring,modular_coloring)
   if not is_proper(g,modular_coloring):
      return(False,'NOT PROPER',coloring,modular_coloring)

   return(True,'SAT',coloring,modular_coloring)
\end{pyblock}

\subsection{Iteration over all colorings}

The Pycosat solver makes it possible to iterate over all solutions, using the
\pyv|itersolve| function.  We will use this to attempt to find all colorings of
a given graph that produce a proper closed modular coloring.  For graphs that
have a large number of such colorings, this can be rather lengthy process, and,
especially for larger graphs, will require a large amount of memory.  We will
therefore provide an option to limit the number of coloring found.  We will
also provide an option to check if the resulting colorings are proper and
closed modular.

\begin{pyblock}
def g_all_pcmc(g,k,limit=0,check=True):
   l, c_m, m_c_m = graph_clauses(g,k)
   if limit==0:
      results = list(pycosat.itersolve(l))
   else:
      results = list(islice(pycosat.itersolve(l),limit))
   
   colorings = [extract_colors(g,c_m,r) for r in results]
   modular_colorings = [extract_colors(g,m_c_m,r) for r in results]

   combined = zip(colorings, modular_colorings)

   return [(c,m) for c,m in combined if (not check) or 
      (is_modular(g,c,m,k) and is_proper(g,m))]
\end{pyblock}


\section{Try it on $K_5$}

Since $K_5$ has true twins, and since we do not check for true
twins, there will not be a proper modular coloring of $K_5$. 

\begin{pyblock}
g = gt.Graph(directed=False)

vs = [g.add_vertex() for i in range(5)]
for v in vs:
   for w in vs:
      if v != w:
         g.add_edge(v,w)

res = g_has_pcmc(g,3)

if not res[0]:
   print(res[1])
else:
   print('Modular coloring is proper')
\end{pyblock}

The result was \printpythontex.

\section{Grid Graphs}

We will see if the solver can find closed modular colorings for
some grid graphs. 

We will start by some imports and some code for properly laying
out grid graphs. Grid graphs are available in graph-tool in the
\pyv|graph_tool.generation| module, and are called 'lattice'
graphs.  Automatic layout algorithms do not work well for them, so
we define our own layout.
\begin{pyblock}
from graph_tool.generation import lattice  

def grid_layout(g,rows,cols):
   grid_pos = g.new_vertex_property("vector<float>")

   for i in range(cols):
      for j in range(rows):
         grid_pos[g.vertex(rows*i + j)] = [i,j]

   return grid_pos
\end{pyblock}

The following function will receive a list of pairs $(m,n)$ and an integer $k$,
and return a dictionary containing, for each pair $(m,n)$, the result of
\pyv|g_has_pcmc| for the grid graph $G_{m,n}$. It will only return the Boolean
result and the string result, without the actual colorings. 
\begin{pyblock}
def check_grids(pairs,k):
   return {p:g_has_pcmc(lattice(p),k)[0:2] for p in pairs}
\end{pyblock}

Since the graphs $G_{m,n}$ and $G_{n,m}$ are equivalent from our point of view,
we only need to check $G_{m,n}$ for $n\le m$.  The following function will
generate all such pairs:
\begin{pyblock}
def triangle(K):
   return ((i,j) for i in range(2,K) for j in range(2,i+1))
\end{pyblock}

The following function will then check all the grid graphs $G_{m,n}$ for $2 \le
m,n < K$ (using the above mentioned symmetry), and print the results.
\begin{pyblock}
def check_all_grids(K,k):
   def format_g(pair):
      return '$G_{{{:d},{:d}}}$'.format(*pair)

   results = check_grids(triangle(K),k)
   graphs_without = [p for p in results if not results[p][0]]
   if len(graphs_without) == 0:
      print('All of the graphs have proper {:d}-modular coloring'.format(k))
   else:
      print('Proper {:d}-modular coloring not found for '.format(k) + 
         ', '.join(format(g) for g in sorted(graphs_without)))
\end{pyblock}

\subsection{$k = 3$}

First we will look at grid graphs $G_{m,n}$ for $2 \le m,n < 21$:
\begin{pyblock}
check_all_grids(21,3)
\end{pyblock}
\printpythontex

\subsection{$k = 2$}

Now we will try with $k=2$:
\begin{pyblock}
check_all_grids(21,2)
\end{pyblock}

\printpythontex

\subsection{All Colorings for $3\times m$}
We will look at coloring by 2 colors for $3\times m$ graphs.

\begin{pyblock}
def all_for_grid(m,n,k,color_names=['black','white']):
   g = lattice([m,n])
   results = g_all_pcmc(g,k)
   print('There are {} proper closed modular colorings mod {} for the graph $G_{{{:d},{:d}}}$.'.format(len(results),k,m,n))

   for i,r in enumerate(results):
      filename = 'grid_{:d}_{:d}_{:d}_{:d}.pdf'.format(m,n,k,i)
      plot_graph(g,r[0],r[1],
                    color_names,
                    grid_layout(g,m,n),
                    filename,size=(50*(n-1),50*m))
      frm = 'width=\\linewidth' if n > 10 else ''
      print('\n\n\\includegraphics[' + frm + ']{' + filename + '}\n\n')
\end{pyblock}

\printpythontex

\begin{pyblock}
for n in range(2,21):
   print('\n\\subsection{{Colorings for $G_{{3,{:d}}}$}}'.format(n))
   all_for_grid(3,n,2)
\end{pyblock}

\printpythontex

\subsection{All Colorings for $4\times m$}
The same code can be used to generate all coloring by 2 colors for $4\times m$ graphs, 
for $m \le 20$:

\begin{pyblock}
for n in range(2,21):
   print('\n\\subsection{{Colorings for $G_{{4,{:d}}}$}}'.format(n))
   all_for_grid(4,n,2)
\end{pyblock}

\printpythontex

\subsection{All Colorings for $5\times m$}
The same code can be used to generate all coloring by 2 colors for $5\times m$ graphs, 
for $m \le 40$:

\begin{pyblock}
for n in range(2,41):
   print('\n\\subsection{{Colorings for $G_{{5,{:d}}}$}}'.format(n))
   all_for_grid(5,n,2)
\end{pyblock}

\printpythontex

\subsection{All Colorings for $6\times m$}
The same code can be used to generate all coloring by 2 colors for $6\times m$ graphs, 
for $m \le 40$:

\begin{pyblock}
for n in range(2,41):
   print('\n\\subsection{{Colorings for $G_{{6,{:d}}}$}}'.format(n))
   all_for_grid(6,n,2)
\end{pyblock}

\printpythontex

\subsection{A Matrix}

Next we will calculate how many closed 2-modular colorings are there for grid
graphs $G_{m,n}$ for $2 \le m,n < 18$.  Since the calculation is very long for
graphs that have large number of such colorings, we will stop after finding 33
colorings for each graph.  We will list these as ``$>32$'' in the table. 

\begin{pyblock}
k = 2
L = 18
M = [[0]*(L-2) for i in range(2,L)]
for m in range(2,L):
   for n in range(2,m+1):
      g = lattice([m,n])
      l, c_m, m_c_m = graph_clauses(g,k)
      M[m-2][n-2] = len(list(islice(pycosat.itersolve(l),33)))
      M[n-2][m-2] = M[m-2][n-2]
\end{pyblock}

\begin{pyblock}
print('\\begin{tabular}{r|' + 'c'*(L-2) + '}\n')
print('&' + '&'.join(str(i) for i in range(2,L)) + '\\\\\\hline\n')
for i,R in enumerate(M):
   print(str(i+2) + '&' + '&'.join(str(j) if j < 33 else '$>32$' for j in R) + '\\\\\n')
print('\\end{tabular}')
\end{pyblock}

\printpythontex

\subsection{All Colorings for $10\times m$}

\begin{pyblock}
for n in range(2,41):
   print('\n\\subsection{{Colorings for $G_{{10,{:d}}}$}}'.format(n))
   all_for_grid(10,n,2)
\end{pyblock}

\printpythontex

\subsection{All Colorings for $14\times m$}

\begin{pyblock}
for n in range(2,41):
   print('\n\\subsection{{Colorings for $G_{{14,{:d}}}$}}'.format(n))
   all_for_grid(14,n,2)
\end{pyblock}

\printpythontex
\end{document}
